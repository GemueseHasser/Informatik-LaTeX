% preamble
\documentclass[12pt,a4paper]{article}

% packages
\usepackage[ngerman]{babel}
\usepackage[utf8]{inputenc}
\usepackage[T1]{fontenc}
\usepackage{amsmath}

% set document properties
\setlength{\textwidth}{160mm}
\setlength{\textheight}{260mm}
\setlength{\evensidemargin}{27mm}
\setlength{\oddsidemargin}{27mm}
\setlength{\topmargin}{0mm}
\setlength{\hoffset}{-1in}
\setlength{\voffset}{-1in}
\setlength{\headheight}{15pt}

% document
\begin{document}
    \setlength{\parindent}{0mm}
    \setcounter{tocdepth}{4}
    \pagestyle{empty}

    \underline{Gelöst werden soll das Integral der Funktion $f(x) = \cos^2(x)$:}

    \vspace{1cm}
    \begin{equation*}
        \int f(x) \, dx = \int \cos^2(x) \, dx
    \end{equation*}

    Das Integral wird mittels partieller Integration umgeschrieben:
    \begin{equation*}
        \int \cos^2(x) \, dx = \sin(x)\cos(x) + \int \sin^2(x) \, dx
    \end{equation*}

    Der trigonometrische Pythagoras wird angewendet, um das Ergebnis umzuschreiben, welches durch die partielle Integration entstanden ist:
    \begin{equation*}
        \sin(x)\cos(x) + \int \sin^2(x) \, dx = \sin(x)\cos(x) - \int \cos^2(x) \, dx + \int 1 \, dx
    \end{equation*}

    Also gilt:
    \begin{alignat*}{3}
        & \int \cos^2(x) \, dx &&= \sin(x)\cos(x) - \int \cos^2(x) \, dx + \int 1 \, dx && \qquad\Big / + \int \cos^2(x) \, dx \\
        \Leftrightarrow\quad & 2 \int \cos^2(x) \, dx &&= \sin(x)\cos(x) + \int 1 \, dx && \qquad\Big / \div 2 \\
        \Leftrightarrow\quad & \int \cos^2(x) \, dx &&= \frac{\sin(x)\cos(x) + x}{2}
    \end{alignat*}


\end{document}