%! Author = jonas
%! Date = 06.02.23

% preamble
\documentclass[12pt,a4paper]{article}

% packages
\usepackage[ngerman]{babel}
\usepackage[utf8]{inputenc}
\usepackage[T1]{fontenc}
\usepackage{amsmath}
\usepackage{hyperref}
\usepackage{hhline}
\usepackage{array}

% set document properties
\setlength{\textwidth}{160mm}
\setlength{\textheight}{260mm}
\setlength{\evensidemargin}{27mm}
\setlength{\oddsidemargin}{27mm}
\setlength{\topmargin}{0mm}
\setlength{\hoffset}{-1in}
\setlength{\voffset}{-1in}
\setlength{\headheight}{15pt}

% document
\begin{document}
    \setlength{\parindent}{0mm}
    \setcounter{tocdepth}{4}
    \pagestyle{empty}

    \underline{Aufgabe 1:}
    
    \begin{table}[h]
        \begin{tabular}[h]{|*{6}{c|}}
            \hline
            & Montag & Dienstag & Mittwoch & Donnerstag & Freitag \\
            \hline
            1. & Religion & Informatik & Religion & Musik & Musik \\
            \hline
            2. & Religion & Informatik & Latein & Musik & Physik \\
            \hline
            3. & Mathe & Geschichte & Geschichte & Mathe & Mathe \\
            \hline
            4. & Mathe & Geschichte & Geschichte & Mathe & Geschichte \\
            \hline
            5. & Physik & Chemie & Deutsch & -- & Latein \\
            \hline
            6. & Physik & Chemie & Deutsch & -- & Latein \\
            \hline
            7. & Informatik & -- & -- & Deutsch & -- \\
            \hline
            8. & Chemie & -- & -- & -- & -- \\
            \hline
            9. & -- & Sport & -- & -- & -- \\
            \hline
            10. & -- & Sport & -- & -- & -- \\
            \hline
        \end{tabular}

        \caption{Stundenplan}
        \label{tab:stundenplan}
    \end{table}

    \underline{Aufgabe 2:}
    
    \begin{table}[h]
        \begin{tabular}[h]{|*{6}{>{$}c<{$}|}}
            \hline
            Baryon & Quark Content & Charge & Mass & Lifetime & Principal Decays \\
            \hline
            p & uud & +1 & 938.272 & \infty & - \\
            n & udd & 0 & 939.565 & 885.7 & pe \bar{v_e} \\
            \Lambda & uds & 0 & 1115.68 & 2.63 \times 10^{-10} & p \pi^-, n \pi^0 \\
            \Sigma^+ & uus & +1 & 1189.37 & 8.02 \times 10^{-11} & p \pi^0, n \pi^+ \\
            \Sigma^0 & uds & 0 & 1192.64 & 7.4 \times 10^{-20} & \Lambda\gamma \\
            \Sigma^- & dds & -1 & 1197.45 & 1.48 \times 10^{-10} & n \pi^- \\
            \Xi^0 & uss & 0 & 1314.8 & 2.90 \times 10^{-10} & \Sigma\pi^0 \\
            \Xi^- & dss & -1 & 1321.3 & 1.64 \times 10^{-10} & \Sigma\pi^- \\
            \hline
        \end{tabular}

        \caption{\textbf{Baryons (spin 1/2)}}
        \label{tab:table}
    \end{table}

    \underline{Aufgabe 3:}

    \begin{table}[h]
        \begin{tabular}[h]{*{2}{c}}
            \hline
            Versuch Nr. & Augenzahl \\
            \hline
            1 & 1 \\
            2 & 3 \\
            3 & 6 \\
            4 & 5 \\
            5 & 3 \\
            6 & 6 \\
            7 & 2 \\
            8 & 1 \\
            9 & 5 \\
            10 & 2 \\
            11 & 2 \\
            12 & 3 \\
            13 & 4 \\
            14 & 6 \\
            15 & 6 \\
            16 & 5 \\
            17 & 5 \\
            18 & 6 \\
            19 & 5 \\
            20 & 4 \\
        \end{tabular}

        \begin{tabular}[h]{*{2}{c}}
            21 & 2 \\
            22 & 2 \\
            23 & 2 \\
            24 & 6 \\
            25 & 3 \\
            26 & 2 \\
            27 & 5 \\
            28 & 6 \\
            29 & 2 \\
            30 & 2 \\
            31 & 4 \\
            32 & 3 \\
            33 & 5 \\
            34 & 2 \\
            35 & 4 \\
            36 & 1 \\
            37 & 1 \\
            38 & 2 \\
            39 & 2 \\
            40 & 4 \\
            41 & 5 \\
            42 & 6 \\
            43 & 1 \\
            44 & 5 \\
            45 & 3 \\
            46 & 6 \\
            47 & 6 \\
            48 & 5 \\
            49 & 5 \\
            50 & 5 \\
            51 & 2 \\
            52 & 5 \\
            53 & 1 \\
            54 & 2 \\
            55 & 5 \\
            56 & 5 \\
            57 & 1 \\
            58 & 6 \\
            59 & 6 \\
            60 & 6 \\
            61 & 3 \\
            62 & 4 \\
            63 & 3 \\
            64 & 3 \\
            65 & 2 \\
            66 & 5 \\
            67 & 5 \\
            68 & 5 \\
            69 & 3 \\
            70 & 5 \\
            71 & 6 \\
            72 & 5 \\
            73 & 6 \\
            74 & 3 \\
            75 & 6 \\
            76 & 1 \\
            77 & 5 \\
            78 & 3 \\
            79 & 4 \\
            80 & 4 \\
            81 & 5 \\
            82 & 5 \\
            83 & 4 \\
            84 & 3 \\
            85 & 4 \\
            86 & 4 \\
            87 & 4 \\
            88 & 1 \\
            89 & 2 \\
            90 & 5 \\
            91 & 6 \\
            92 & 3 \\
            93 & 5 \\
            94 & 1 \\
            95 & 5 \\
            96 & 5 \\
            97 & 2 \\
            98 & 3 \\
            99 & 2 \\
            100 & 4 \\
            \hline
        \end{tabular}

        \caption{Aufgabe 6: Urliste}
        \label{tab:urliste}
    \end{table}

\end{document}
