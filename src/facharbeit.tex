%! Author = jonas
%! Date = 14.02.23

% preamble
\documentclass[12pt,a4paper]{article}

% packages
\usepackage[ngerman]{babel}
\usepackage[utf8]{inputenc}
\usepackage[T1]{fontenc}
\usepackage{hyperref}
\usepackage{listings}
\usepackage{color}
\usepackage{unicode-math}

% set document properties
\setlength{\textwidth}{160mm}
\setlength{\textheight}{260mm}
\setlength{\evensidemargin}{27mm}
\setlength{\oddsidemargin}{27mm}
\setlength{\topmargin}{0mm}
\setlength{\hoffset}{-1in}
\setlength{\voffset}{-1in}
\setlength{\headheight}{15pt}

% define java-code properties
\definecolor{dkgreen}{rgb}{0,0.6,0}
\definecolor{gray}{rgb}{0.5,0.5,0.5}
\definecolor{mauve}{rgb}{0.58,0,0.82}

\lstset{ %
    language=Java,
    basicstyle=\footnotesize,
    numbers=left,
    numberstyle=\tiny\color{gray},
    stepnumber=1,
    numbersep=5pt,
    backgroundcolor=\color{white},
    showspaces=false,
    showstringspaces=false,
    showtabs=false,
    frame=single,
    rulecolor=\color{black},
    tabsize=4,
    captionpos=b,
    breaklines=true,
    breakatwhitespace=falses,
    title=\lstname,
    keywordstyle=\color{blue},
    commentstyle=\color{dkgreen},
    stringstyle=\color{mauve},
    inputencoding=utf8,
    extendedchars=true,
    literate=%
        {Ä}{{\"A}}1
        {Ö}{{\"O}}1
        {Ü}{{\"U}}1
        {ä}{{\"a}}1
        {ö}{{\"o}}1
        {ü}{{\"u}}1
        {ß}{{\ss}}1,
}

% define calculator project path
\newcommand{\project}{/home/jonas/Dokumente/Java-Projekte/GraphingCalculator/src/main/java/de/jonas/graphingcalculator}

% document
\begin{document}
    \setlength{\parindent}{0mm}
    \setcounter{tocdepth}{4}
    \pagestyle{plain}

    \title{Grafikfähiger Taschenrechner \\[1ex] \large - Facharbeit Informatik -}
    \date{}
    \author{Jonas Lobe}

    \maketitle
    \newpage

    \tableofcontents

    \newpage

    Der grafikfähige Taschenrechner wird in der Programmiersprache $Java$ entwickelt und bietet die normale Berchnung
    von Termen ohne Variablen an, das Zeichnen von Funktionen und das Anlegen von Wertetabellen.
    Für alle Eingaben mit Variable wird ausschließlich die Variable $x$ verwendet.
    Um in einer Funktion eine Wurzel zu nutzen, wird die Standard-Abkürzung $sqrt$ genutzt.


    \section{Funktionen des Taschenrechners}

    \subsection{Berechnung von Termen} \label{subsection:calculator}
    Der Taschenrechner ermöglicht das normale Berechnen von Termen ohne Variablen.
    Dabei stehen dem Anwender verschiedene Rechenoperationenen zur Verfügung:

    \begin{itemize}
        \item Standard Rechenoperationen: Addition, Subtraktion, Division und Multiplikation
        \item Trigonometrische Funktionen: Sinus, Cosinus und Tangens
        \item Logarithmus und naürlicher Logarithmus
        \item Wurzel ziehen
        \item Potenzieren
        \item Eulersche Konstante (e)
        \item PI
    \end{itemize}

    \subsection{Zeichnen einer Funktion}
    Bei der Eingabe der Funktionen stehen dem Nutzer dieselben mathematischen Operationen wie bei den Funktionen
    des Taschenrechners $\ref{subsection:calculator}$ zur Verfügung.
    Die Skalierung der Anzeige wird mit einem positiven Wert für die x-Achse und einem für die y-Achse angegeben.
    Die Skalierung wird dann automatisch vorgenommen sowohl im negativen, als auch im positiven Bereich.
    Wählt man also als x-Skalierung den Wert 10 und als y-Skalierung den Wert 15,
    geht die Skalierung auf der x-Achse von -10 bis 10 und auf der y-Achse von -15 bis 15.
    Für die Skalierung der Koordinaten-Achsen dürfen ausschließlich natürliche Zahlen verwendet werden.\par

    \newpage

    Wenn die Funktion gezeichnet wurde, stehen dem Anwender folgende Optionen unter fogenden Menüs zur Verfügung:

        {
        \renewcommand{\labelenumi}{\bf \Alph{enumi}}
        \renewcommand{\labelenumii}{\arabic{enumii}}

        \begin{enumerate}
            \item Punkte
            \begin{enumerate}
                \item Nullstellen anzeigen
                \item Extremstellen anzeigen
                \item Wendepunkte anzeigen
                \item Sattelpunkte anzeigen
            \end{enumerate}

            \item Markierungen
            \begin{enumerate}
                \item Punkt einzeichnen
                \item Letzten Punkt entfernen
            \end{enumerate}

            \item Ableitung
            \begin{enumerate}
                \item erste Ableitung
                \item zweite Ableitung
                \item dritte Ableitung
            \end{enumerate}
            \item Extra
            \begin{enumerate}
                \item Tangente anlegen
            \end{enumerate}
        \end{enumerate}
    }

    Außerdem wird der Punkt an der Stelle der Maus angezeigt, solange die Maus gedrückt gehalten wird.
    Wenn man die Maus bewegt, aktualisiert sich der Punkt, der angezeigt wird automatisch und wird neu eingezeichnet.
    Bei den Nullstellen wird ausschließlich die x-Koordinate eingezeichnet, um Platz zu sparen.
    Beim Einzeichnen der Tangente wird zudem die Gleichung der Tangente dargestellt.

    \subsection{Anlegen einer Wertetabelle}
    Der Taschenrechner ermöglicht es dem Anwender eine Wertetabelle zu einer Funktion in einem bestimmten Intervall anzulegen.
    Dieses Intervall wird gewählt durch x-min und x-max, wodurch der kleinste und der größte x-Wert in der Wertetabelle definiert wird.
    Für diese Werte dürfen ausschließlich ganze Zahlen genutzt werden.\par
    Die Schrittweite kann auch definiert werden; hierbei dürfen alle positiven reelen Zahlen gewählt werden.

    \newpage


    \section{Programmiertechnische Details}

    In der Main-Klasse dieses Taschenrechners wird ausschließlich eine Instanz eines Haupt-Fensters erzeugt und geöffnet.
    Dieses Projekt ist ein Maven-Projekt und muss somit auch mit Maven kompiliert werden.
    Dieses Projekt hat zwei Maven Abhängigkeiten:

    \begin{enumerate}
        \item Lombok
        \item JetBrains-Annotations
    \end{enumerate}

    Mithilfe von Lombok können Konstruktoren mit einem Parameter für jede Variable, die eine Initialisierung benötigt,
    erzeugt werden.
    Zudem können Getter und Setter durch einfache Anmekungen (Annotations) im Code beim Prozess des Kompilierens mit
    Lombok erzeugt werden.

    \subsection{Erzeugung eines Fensters}
    Um ein Fenster zu erzeugen, gibt es die abstrakte Klasse Gui.java, welche eine Instanz eines JFrame darstellt.
    Jedes Fenster des Taschenrechners wird mittels eines Guis erzeugt, welches Titel und Größe übergeben bekommt.
    Damit werden dann die grundlegenden Eigenschaften des Fensters festgelegt.

    \subsubsection{Das Haupt-Fenster}
    Das Haupt-Fenster stellt ein Menü mit allen Aktionen, die mit dem Taschenrechner möglich sind, zur Verfügung.
    Dieses Gui wird auch als allererstes in der main-Methode der Anwendung instanziiert und geöffnet.
    Ausgehend von diesem Fenster lassen sich dann die weiteren Fenster für den Taschenrechner,
    das Zeichnen eine Funktion und das Anlegen einer Wertetabelle öffnen.

    \subsubsection{Taschenrechner}
    Das Taschenrechner-Fenster öffnet dem Anwender eine Oberfläche, auf der man einen Term eingeben und berechnen lassen kann.
    Diesen Term kann man mithilfe der Knöpfe in dem Fenster eingeben; man kann diese Knöpfe aber auch mit den Tasten
    auf der Tastatur drücken.

    \subsubsection{Funktion zeichnen}
    Bevor sich das Fenster mit der gezeichneten Funktion öffnet, muss der Nutzer in einem JOptionPane die Funktion
    und deren Skalierung eingeben.
    Mithilfe des Function-Handlers werden dann in kleinen Abständen alle Funktions-Werte ausgerechnet in der
    entsprechenden Skalierung und in das Koordinatensystem eingezeichnet und miteinander verbunden.\par
    Außerdem wird bei jedem Zeichnen geprüft, welche Werte auf true stehen und welche Punkte/Funktionen somit
    eingezeichnet werden sollen.

    \subsubsection{Wertetabelle}
    Bevor sich das Fenster mit der Wertetabelle öffnet, muss der Nutzer in einem JOptionPane die Funktion, das Intervall,
    in dem die Wertetabelle angelegt werden soll und die Schrittweite definieren.
    Dann öffnet sich das Fenster mit der entsprechenden Wertetabelle.
    Hierbei sind die Zeilen der Übersichtlichkeit halber abwechselnd grau und weiß.

    \subsection{Enum-Konstanten}
    In dem CalculatorActionType.java werden alle Aktionen gespeichert, die der Taschenrechner ausführen kann.
    Diese Typen werden in dem Haupt-Fenster genutzt, um die einzelnen Aktionen aufzuführen.

    \subsection{Handler}

    \subsubsection{Function-Handler}
    Mithilfe des Function-Handlers lassen sich in einem bestimmten Intervall in kleinen Abständen alle
    Funktionswerte berechnen und zurückgeben.
    Außerdem bietet der Function-Handler die utility-Methode eval(), mit der sich ein Term ausrechnen lässt.\par
    Zusätzlich können mit dem Function-Handler alle Nullstellen, Extremstellen, Wendepunkte und Sattelpunkte
    in einer Map zurückgegeben werden.
    Auch die Funktion einer Tangente an einer gewissen Stelle $x$ kann zurückgegeben werden.

    \subsubsection{File-Handler}
    Mithilfe des File-Handlers kann dem Nutzer ein Menü geöffnet werden, worin eine betimmte Datei gewählt werden kann.
    Diese Datei wird dann zurückgegeben.

    \newpage


    \section{Source-Code}
    Der Source-Code ist grundsätzlich auf GitHub unter\par
    https://github.com/GemueseHasser/GraphingCalculator\par
    zu finden.

    \vspace{1cm}

    \subsection{Main-Klasse}

    \lstinputlisting[language=Java, caption=GraphingCalculator.java]{\project/GraphingCalculator.java}

    \newpage
    \subsection{Grafische Oberflächen}

    \subsubsection{Haupt-Fenster}
    \lstinputlisting[language=Java, caption=MainGui.java]{\project/gui/MainGui.java}

    \newpage

    \subsubsection{Taschenrechner}
    \lstinputlisting[language=Java, caption=CalculatorGui.java]{\project/gui/CalculatorGui.java}

    \newpage

    \subsubsection{Funktion zeichnen}
    \lstinputlisting[language=Java, caption=FunctionGui.java]{\project/gui/FunctionGui.java}

    \newpage

    \subsubsection{Wertetabelle}
    \lstinputlisting[language=Java, caption=ValueTableGui.java]{\project/gui/ValueTableGui.java}

    \newpage

    \subsection{Enum-Konstanten}

    \subsubsection{Calculator-Action-Type}
    \lstinputlisting[language=Java, caption=CalculatorActionType.java]{\project/constant/CalculatorActionType.java}

    \newpage

    \subsection{Handler}

    \subsubsection{Function-Handler}
    \lstinputlisting[language=Java, caption=FunctionHandler.java]{\project/handler/FunctionHandler.java}

    \newpage

    \subsubsection{FileHandler}
    \lstinputlisting[language=Java, caption=FileHandler.java]{\project/handler/FileHandler.java}

    \newpage
    
    \subsection{Objekte}

    \subsubsection{Gui-Superklasse}
    \lstinputlisting[language=Java, caption=Gui.java]{\project/object/Gui.java}

    \newpage
    
    \subsubsection{Zeichen-Objekt für das Zeichnen einer Funktion}
    \lstinputlisting[language=Java, caption=DrawFunction.java]{\project/object/DrawFunction.java}

    \newpage

    \subsubsection{Taschenrechner-Feld}
    \lstinputlisting[language=Java, caption=CalculatorField.java]{\project/object/CalculatorField.java}

    \newpage

    \subsubsection{Ableitung}
    \lstinputlisting[language=Java, caption=Derivation.java]{\project/object/Derivation.java}

    \newpage

    \subsubsection{Punkt}
    \lstinputlisting[language=Java, caption=Point.java]{\project/object/Point.java}
\end{document}
