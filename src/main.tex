%! Author = jonas
%! Date = 06.02.23

% preamble
\documentclass[12pt,a4paper]{article}

% packages
\usepackage[ngerman]{babel}
\usepackage[utf8]{inputenc}
\usepackage[T1]{fontenc}

% set document properties
\setlength{\textwidth}{160mm}
\setlength{\textheight}{260mm}
\setlength{\evensidemargin}{27mm}
\setlength{\oddsidemargin}{27mm}
\setlength{\topmargin}{0mm}
\setlength{\hoffset}{-1in}
\setlength{\voffset}{-1in}
\setlength{\headheight}{15pt}

% document
\begin{document}
    \setlength{\parindent}{0mm}
    \setcounter{tocdepth}{4}
    \pagestyle{empty}
    \large

    Schreibe folgenden Text über die Bruchrechnung.
    \begin{itemize}
        \item Ein Bruch behält seinen Wert, wenn man Zähler und Nenner mit der gleichen Zahl $\delta \neq 0$ multipliziert.
        Man nennt das $Erweitern$.
        \par $\frac{a}{b} = \frac{a \cdot \delta}{b \cdot \delta}$.

        \item Haben zwei Brüche denselben Nenner, so werden sie addiert (subtrahiert), indem man die Zähler addiert (subtrahiert).
        \par $\frac{a}{b} + \frac{c}{b} = \frac{a + c}{b}$ \hspace{1cm} $\frac{a}{b} - \frac{c}{b} = \frac{a - c}{b}$

        \item Zwei Brüche werden multipliziert, indem man die Zähler und die Nenner jeweils miteinander multipliziert.
        \hspace{1cm} $\frac{a}{b} \cdot \frac{c}{d} = \frac{a \cdot c}{b \cdot d}$

        \item Zu einem Bruch nennt man den Bruch mit vertauschtem Zähler und Nenner den Kehrwert.
        Man dividiert einen Bruch durch einen anderen Bruch, indem man ihn mit dessen Kehrwert multipliziert.
        \par $\frac{a}{b} : \frac{c}{d} = \frac{a}{b} \cdot \frac{d}{c}$
    \end{itemize}
\end{document}
